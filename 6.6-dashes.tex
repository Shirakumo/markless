\definesubsection{Dashes}
\defineidentifier{dash}{---?}
\definetextualcomponent{en-dash}{display: en-dash}
\definetextualcomponent{em-dash}{display: em-dash} \\

The dash is an \g{entity inline directive}. The \g{resulting textual component} is the \inline$en-dash$ if the \g{identifier} matches two hyphens, and \inline$en-dash$ if it matches three hyphens. If the \g{document} does not have direct support for dashes, a fallback character may be used when appropriate instead. In unicode encoded documents, this should be \unicode{2013} for the en-dash and \unicode{2014} for the em-dash. \\

\begin{examples}
  \example{A game -- or gamble --- if you will.}{A game -- or gamble --- if you will.}
\end{examples}

%%% Local Variables:
%%% mode: latex
%%% TeX-master: "markless"
%%% TeX-engine: luatex
%%% TeX-command-extra-options: "-shell-escape"
%%% End:
