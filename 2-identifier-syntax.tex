\definesection{Identifier Syntax}
In order to concisely specify \gpl{identifier} we use a special syntax, of which the full grammar and semantics are reflected here using BNF notation. \\

\begin{tabular}{lcl}
  \definesyntax{rule} &::=& ``\inline{(}''? (\syntax{matcher}* \syntax{quantifier}?)+ ``\inline{)}''? \\
  \definesyntax{matcher} &::=& \syntax{string} | \syntax{some-characters} | \syntax{any-character} \\
  \definesyntax{string} &::=& \syntax{character}+ \\
  \definesyntax{some-characters} &::=& ``\inline{[}'' \syntax{character}+ ``\inline{]}'' \\
  \definesyntax{any-character} &::=& ``\inline{.}'' \\
  \definesyntax{quantifier} &::=& \syntax{one-or-more} | \syntax{none-or-more} | \syntax{one-or-none} | \syntax{repeat} \\
  \definesyntax{one-or-more} &::=& \syntax{rule} ``\inline{+}'' \\
  \definesyntax{none-or-more} &::=& \syntax{rule} ``\inline{*}'' \\
  \definesyntax{one-or-none} &::=& \syntax{rule} ``\inline{?}'' \\
  \definesyntax{repeat} &::=& \syntax{rule} ``\inline${$'' \syntax{number} (``\inline{,}'' \syntax{number}?)? ``\inline$}$'' \\
  \definesyntax{binding} &::=& ``\inline{<}'' \syntax{name} \syntax{rule} ``\inline{>}'' \\
  \definesyntax{binding-reference} &::=& ``\inline{<}'' \syntax{name} ``\inline{>}'' \\
  \definesyntax{name} &---& Some \g{alphanumeric} \g{string} to identify the text matched by the \syntax{rule}. \\
  \definesyntax{character} &---& A \g{character}. \\
  \definesyntax{number} &---& An integer. \\
\end{tabular} \\

\noindent Appearing within the ``'' quotes are \gpl{character} to be found in the \g{identifier specifier}. 

\noindent If a backslash appears anywhere within the \g{identifier specifier}, it is ignored and the \g{character} immediately after it is taken literally without being interpreted as one of the \gpl{character} in the syntax rules and without being interpreted using this backslash rule. Thus two backslashes immediately after one another are interpreted as a single, literal backslash \g{character}.

\noindent In order for a \syntax{string} to \g{match}, the exact sequence of \gpl{character} must be found.

\noindent In order for \syntax{some-characters} to \g{match}, one of the \gpl{character} must be found.

\noindent In order for \syntax{any-character} to \g{match}, a single \g{character} must be found, but it matters not which \g{character} it is.

\noindent In order for \syntax{one-or-more} to \g{match}, the \syntax{rule} must be \glink{match}{matched} at least once, but may be \glink{match}{matched} an arbitrary number of times immediately after each other. The \syntax{rule} is only repeatedly \glink{match}{matched} until the \syntax{rule} immediately after the \syntax{one-or-more} is \glink{match}{matched}.

\noindent In order for \syntax{none-or-more} to \g{match}, the \syntax{rule} does not have to be \glink{match}{matched} at all, but may be \glink{match}{matched} an arbitrary number of times immediately after each other. The \syntax{rule} is only repeatedly \glink{match}{matched} until the \syntax{rule} immediately after the \syntax{none-or-more} is \glink{match}{matched}.

\noindent In order for \syntax{one-or-none} to \g{match}, the \syntax{rule} does not have to be \glink{match}{matched} at all, but if it is, it is only \glink{match}{matched} exactly once.

\noindent In order for \syntax{repeat} to \g{match}, the \syntax{rule} must be \glink{match}{matched} at least the number of times of the first \syntax{number}. If only the comma is provided after the first \syntax{number}, it may \g{match} any number of times greater than the first \syntax{number}. If the second \syntax{number} is provided, it cannot \g{match} more times than the second \syntax{number}.

\noindent In order for a \syntax{binding} to \g{match}, the \syntax{rule} contained must \g{match}. The specific \g{string} \glink{match}{matched} by the \syntax{rule} is then associated with the \syntax{name} of the \syntax{binding}.

\noindent In order for a \syntax{binding-reference} to \g{match}, the exact \g{string} associated with the \syntax{name} of the \syntax{binding} must be found.

%%% Local Variables:
%%% mode: latex
%%% TeX-master: "0-markless"
%%% End:
