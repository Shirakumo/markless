\definesubsection{Alignment}
\begin{identifier}{Left Align}
  \|\<<content .*>
\end{identifier}
\begin{identifier}{Right Align}
  \|\><content .*>
\end{identifier}
\begin{identifier}{Center}
  \>\<<content .*>
\end{identifier}
\begin{identifier}{Justify}
  \|\|<content .*>
\end{identifier}
\definetextualcomponent{left align}{text-align: left}
\definetextualcomponent{right align}{text-align: right}
\definetextualcomponent{center}{text-align: center}
\definetextualcomponent{justify}{text-align: justify} \\

All alignment directives are \g{spanning line directive} that change the alignment of a body of \g{text}. The alignment content can contain any \g{directive} with the condition that the \gpl{directive} are matched against the \g{text} of the \g{resulting textual component}. \\

\begin{examples}
  \begin{examplesource}
    |< Left
    >< Center
    |> Right
  \end{examplesource}
  \begin{exampleoutput}
    \begin{minipage}{0.5\textwidth}
      \begin{flushleft}Left\end{flushleft}
      \begin{center}Center\end{center}
      \begin{flushright}Right\end{flushright}
    \end{minipage}
  \end{exampleoutput}
\end{examples}

%%% Local Variables:
%%% mode: latex
%%% TeX-master: "markless"
%%% TeX-engine: luatex
%%% TeX-command-extra-options: "-shell-escape"
%%% End:
