\definesubsection{Italic}
\defineidentifier{italic}{/<content .*>/}
\definetextualcomponent{italic}{font-style: italic} \\

Italic is a \g{surrounding inline directive}. It marks the \g{text} to belong to a \g{textual component} that sets the style of the font to italic. Only the \g{text} held by the \g{content binding} is outputted to the \g{resulting textual component}. \\

\begin{examples}
  \example{I /really/ don't care.}{I \textit{really} don't care.}
  \example{/call\\/cc/ is important.}{\textit{call/cc} is important.}
\end{examples}

%%% Local Variables:
%%% mode: latex
%%% TeX-master: "markless"
%%% TeX-engine: luatex
%%% TeX-command-extra-options: "-shell-escape"
%%% End:
