\definesubsection{Instruction}
\begin{identifier}{instruction}
\! <instruction .*>
\end{identifier}

The instruction is a \g{singular line directive}. Its purpose is to interact with the \g{implementation} and cause it to perform differently. There is no corresponding \g{resulting textual component} for the instruction \g{directive}. \\

An \g{implementation} is allowed to add further instructions. If an instruction is not recognised, the \g{implementation} must \glink{signalling}{signal} an \g{error}. \\

\definesubsubsection{Set}
\defineinstruction{set}{set <variable ![ ]+> <value .+>} \\

Sets the state of the \g{variable} of the given name to a certain value. An \g{implementation} may check the value for validity and \glink{signalling}{signal} an \g{error} if it is invalid.  An \g{implementation} is allowed to add further variables. If a variable is not recognised, the \g{implementation} must \glink{signalling}{signal} an \g{error}. \\

\definesubsubsubsection{Line Break Mode}
\definevariable{line-break-mode}{show} \\

The \var{line-break-mode} variable may only assume two values: \inline$show$, and \inline$hide$. If the line break mode is \inline$show$, when the processor encounters an unescaped \g{newline}, a new \g{line} is started in the output \g{document}. \\

\begin{examples}
\begin{examplesource}
! set line-break-mode show
foo
bar\
baz
! set line-break-mode hide
bada
boom
\end{examplesource}
  \begin{exampleoutput}
    foo\\
    barbaz\\
    badaboom
  \end{exampleoutput}
\end{examples}

%%% Local Variables:
%%% mode: latex
%%% TeX-master: "markless"
%%% TeX-engine: luatex
%%% TeX-command-extra-options: "-shell-escape"
%%% End:
\definesubsubsubsection{Metadata}
\definevariable{author}{} \\
\definevariable{copyright}{} \\

Declares \g{metadata} about the \g{document}. The \g{implementation} may use this information and embed it into the output \g{document}.

%%% Local Variables:
%%% mode: latex
%%% TeX-master: "markless"
%%% TeX-engine: luatex
%%% TeX-command-extra-options: "-shell-escape"
%%% End:

%%% Local Variables:
%%% mode: latex
%%% TeX-master: "markless"
%%% TeX-engine: luatex
%%% TeX-command-extra-options: "-shell-escape"
%%% End:
\definesubsubsection{Message}
\defineinstruction{info}{info <message .*>}
\defineinstruction{warn}{warn <message .*>}
\defineinstruction{error}{error <message .*>} \\

The \instr{info} instruction causes the \g{implementation} to \glink{signalling}{signal} the given message.
The \instr{warn} instruction causes the \g{implementation} to \glink{signalling}{signal} a \g{warning} with the given message.
The \instr{error} instruction causes the \g{implementation} to \glink{signalling}{signal} an \g{error} with the given message. \\

%%% Local Variables:
%%% mode: latex
%%% TeX-master: "markless"
%%% TeX-engine: luatex
%%% TeX-command-extra-options: "-shell-escape"
%%% End:
\definesubsubsection{Include}
\defineinstruction{include}{include <file .*>} \\

Causes the implementation to \glink{interpretation}{interpret} the contents of the given file. If the file is not accessible for some reason, the \g{implementation} must \glink{signalling}{signal} an \g{error}. \\

%%% Local Variables:
%%% mode: latex
%%% TeX-master: "markless"
%%% TeX-engine: luatex
%%% TeX-command-extra-options: "-shell-escape"
%%% End:
\definesubsubsection{Directives}
\defineinstruction{disable}{disable <directive ![ ]+>( <directive ![ ]+>)*}
\defineinstruction{enable}{enable <directive ![ ]+>( <directive ![ ]+>)*} \\

The \instr{disable} and \instr{enable} instructions cause the \g{implementation} to respectively disable or enable the named \gpl{directive}. The name of a directive must be recognised regardless of the case the user writes the directive in. If a given name is not recognised, the \g{implementation} may \glink{signalling}{signal} a \g{warning}. If a user tries to disable the \comp{paragraph} \g{directive}, an \g{error} must be \glink{signalling}{signalled}. \\

\begin{examples}
\begin{examplesource}
! disable instruction
! error Exit!
\end{examplesource}
  \begin{exampleoutput}
    ! error Exit!
  \end{exampleoutput}
\end{examples}

%%% Local Variables:
%%% mode: latex
%%% TeX-master: "markless"
%%% TeX-engine: luatex
%%% TeX-command-extra-options: "-shell-escape"
%%% End:


%%% Local Variables:
%%% mode: latex
%%% TeX-master: "markless"
%%% TeX-engine: luatex
%%% TeX-command-extra-options: "-shell-escape"
%%% End: