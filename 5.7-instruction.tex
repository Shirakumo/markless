\definesubsection{Instruction}
\begin{identifier}{instruction}
\! <instruction .*>
\end{identifier}

The instruction is a \g{singular line directive}. Its purpose is to interact with the \g{implementation} and cause it to perform differently. There is no corresponding \g{resulting textual component} for the instruction \g{directive}. \\

An \g{implementation} is allowed to add further instructions. If an instruction is not recognised, the \g{implementation} must \glink{signalling}{signal} an \g{error}. \\

\input{5.7.1-set.tex}
\input{5.7.2-message.tex}
\input{5.7.3-include.tex}
\definesubsubsection{Directives}
\defineinstruction{disable}{disable <directive ![ ]+>( <directive ![ ]+>)*}
\defineinstruction{enable}{enable <directive ![ ]+>( <directive ![ ]+>)*} \\

The \instr{disable} and \instr{enable} instructions cause the \g{implementation} to respectively disable or enable the named \gpl{directive}. If a given name is not recognised, the \g{implementation} may \glink{signalling}{signal} a \g{warning}. If a user tries to disable the \comp{paragraph} \g{directive}, an \g{error} must be \glink{signalling}{signalled}. \\

\begin{examples}
\begin{examplesource}
! disable instruction
! error Exit!
\end{examplesource}
  \begin{exampleoutput}
    ! error Exit!
  \end{exampleoutput}
\end{examples}

%%% Local Variables:
%%% mode: latex
%%% TeX-master: "markless"
%%% TeX-engine: luatex
%%% TeX-command-extra-options: "-shell-escape"
%%% End:

\definesubsubsection{Label}
\defineinstruction{label}{label <name .*>} \\

Associates a \g{label} with the \g{component} that immediately precedes this \g{instruction} at the current level. The \g{implementation} must \glink{signalling}{signal} an \g{error} if there is no preceding component at the current level. \\

%%% Local Variables:
%%% mode: latex
%%% TeX-master: "markless"
%%% TeX-engine: luatex
%%% TeX-command-extra-options: "-shell-escape"
%%% End:

\definesubsubsection{Raw}
\defineinstruction{raw}{raw <backend ![ ]+> <content .*>} \\

If the \g{implementation}'s chosen output backend matches that of the \inline$backend$ \g{binding}, the \g{implementation} should emit the \inline$content$ \g{binding}'s text verbatim into the resulting document. This should allow creating output specific effects. The exact semantics and results of this are \g{implementation dependant}. If the \g{implementation}'s chosen output backend does not match, the instruction is ignored. \\

Users should note that basic Markless parsing rules such as backslash escapes still apply for the \inline$content$, so the content is not copied directly 1:1 from the source text to the output document. \\

\begin{examples}
  \example{! raw latex \\\\textit\{Hello\}}{\textit{Hello}}
  \example{! raw html <blink>Hello</blink>}{}
\end{examples}

%%% Local Variables:
%%% mode: latex
%%% TeX-master: "markless"
%%% TeX-engine: luatex
%%% TeX-command-extra-options: "-shell-escape"
%%% End:

%%% Local Variables:
%%% mode: latex
%%% TeX-master: "markless"
%%% TeX-engine: luatex
%%% TeX-command-extra-options: "-shell-escape"
%%% End: