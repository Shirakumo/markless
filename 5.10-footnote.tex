\definesubsection{Footnote}
\begin{identifier}{footnote}
  \[<number ~n+>\] <content .+>
\end{identifier}
\definetextualcomponent{footnote}{} \\

The footnote is a \g{singular line directive}. Outputted to the \g{resulting textual component} is the \g{text} held by the \inline$number$ \g{binding} followed by a \unicode{3A}, followed by the \g{text} held by the \g{content binding}. The footnote can only contain \gpl{inline directive}. \\

Unlike other \gpl{directive} the footnote's \g{resulting textual component} cannot be placed where the \g{identifier} is found. It must be placed such that it is at the end of a \g{page} in the \g{document}.\\

The \g{resulting textual component} is associated with a \g{label} with the name being the content of the \inline$number$ \g{binding}. \\

\begin{examples}
  \begin{examplesource}
    Examples[1] are not authoritative.
    
    [1] Examples are things like this.
  \end{examplesource}
  \begin{exampleoutput}
    Examples\raisebox{.4ex}{\scriptsize \hyperref[footnote:ex1]{[1]}} are not authoritative. \\
    \rule{0.2\textwidth}{1pt} \\
    \label{footnote:ex1}1: Examples are things like this.
  \end{exampleoutput}
\end{examples}

%%% Local Variables:
%%% mode: latex
%%% TeX-master: "markless"
%%% TeX-engine: luatex
%%% End: