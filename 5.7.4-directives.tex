\definesubsubsection{Directives}
\defineinstruction{disable}{disable <directive ![ ]+>( <directive ![ ]+>)*}
\defineinstruction{enable}{enable <directive ![ ]+>( <directive ![ ]+>)*} \\

The \instr{disable} and \instr{enable} instructions cause the \g{implementation} to respectively disable or enable the named \gpl{directive}. The name of a directive must be recognised regardless of the case the user writes the directive in. If a given name is not recognised, the \g{implementation} may \glink{signalling}{signal} a \g{warning}. If a user tries to disable the \comp{paragraph} \g{directive}, an \g{error} must be \glink{signalling}{signalled}. \\

\begin{examples}
\begin{examplesource}
! disable instruction
! error Exit!
\end{examplesource}
  \begin{exampleoutput}
    ! error Exit!
  \end{exampleoutput}
\end{examples}

%%% Local Variables:
%%% mode: latex
%%% TeX-master: "markless"
%%% TeX-engine: luatex
%%% TeX-command-extra-options: "-shell-escape"
%%% End:
