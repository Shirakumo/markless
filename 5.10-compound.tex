\definesubsection{Compound}
\defineidentifier{compound-content}{<start ["]+><content !["].*><start>|<content ![ (]+>}
\defineidentifier{compound-options}{\((in|to)( <option .*>,)+\)}
\defineidentifier{compound}{\{compound-content\}\{compound-options\}}
\definetextualcomponent{compound}{} \\

The \comp{compound} \g{directive} determines its \g{style} dynamically by the additive combination of present \ident{compound-options}. In the case where the style combination of two options conflicts, the style of the last option has priority. \\

Only the \g{text} held by the \g{content binding} is outputted to the \g{resulting textual component}. The \ident{compound-options} \g{identifier} cannot contain any other \gpl{directive}.\\

An \g{implementation} must at least support the options specified in this section, but may add additional options the syntax and implications of which are completely \g{implementation dependant}. If an option is found that the \g{implementation} does not support, it is ignored. \\

\input{5.10.1-bold.tex}
\input{5.10.2-italic.tex}
\input{5.10.3-underline.tex}
\input{5.10.4-strikethrough.tex}
\definesubsubsection{Spoiler}
\defineidentifier{compound-spoiler}{spoiler}
\definestyle{compound-spoiler}{display: hidden} \\

If given, this option marks the \g{style} to obscure the \g{text} in such a manner that the \g{user} must perform an \g{action} in order to reveal the \g{text}.\\

\begin{examples}
  \example{This is a secret(in spoiler)!}{This is a \colorbox{black}{secret}!}
\end{examples}

%%% Local Variables:
%%% mode: latex
%%% TeX-master: "0-markless"
%%% TeX-engine: luatex
%%% End:

\input{5.10.6-font.tex}
\input{5.10.7-color.tex}
\input{5.10.8-size.tex}
\input{5.10.9-hyperlink.tex}

%%% Local Variables:
%%% mode: latex
%%% TeX-master: "0-markless"
%%% TeX-engine: luatex
%%% End: