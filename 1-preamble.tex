\definesection{Preamble}
Markless is a new markup standard that focuses on being intuitive and fast to parse. Being a purely text-based markup, no complicated editor software is required to create documents in it. With its focus on intuition and consistency it should also be a good fit as a markup choice for text based platforms such as chat, forums, etc. Markless does not specify its results based on another document format, meaning that an implementation could be written to turn a Markless document into practically any other format. Markless is strict and does not allow for any ambiguities in its markup. This both makes it less confusing for the user, and easier to parse for a program. Being based on a specification rather than a reference implementation, Markless also offers the users a much more stable and reliant source to turn to in case of questions about the behaviour of an implementation.\\

This document specifies the way a Markless \g{document} is treated and how the various markup \gpl{directive} are to be \glink{interpretation}{interpreted}. It does not describe the technological aspects of writing an \g{implementation} for Markless. It should also not be used as a guide or introduction on how to write Markless documents, but rather as a reference if you should want to write a new \g{implementation} or are unsure about the behaviour of an existing one. This document also describes the terminology to allow talking about Markless terms unambiguously. See the \sectionref{glossary} for reference. \\

Included in most sections are one or more examples. These examples exist purely for illustrative purposes and are not normative. An \g{implementation} may deviate from the behaviour illustrated by the examples as long as it adheres to the actual description of this specification.

%%% Local Variables:
%%% mode: latex
%%% TeX-master: "markless"
%%% TeX-engine: luatex
%%% TeX-command-extra-options: "-shell-escape"
%%% End:
