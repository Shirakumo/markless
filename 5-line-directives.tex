\definesection{Line Directives}
In order for a \g{directive} to be a \g{line directive}, its \g{identifier} must \g{match} the beginning of a \g{line} and either the end of a \g{line} or the beginning of a different \g{line}. Thus the \g{identifier} of each \g{line directive} only matches at the beginning of a \g{line}.\\

A \g{textual component} specified by a \g{line directive} can potentially contain any other \g{textual component}. Therefore, any \g{directive} is potentially recognisable within a \g{line directive}, including other \gpl{line directive}. However, a \g{line directive} may explicitly restrict which \gpl{directive} are recognised within itself. A \g{line directive} cannot cross the boundaries of another \g{line directive} of a different kind. If such a case were to occur, the current \g{line directive} is forcibly ended without regard for any possible trailing \g{match}.

\definesubsubsection{Singular Line Directives}
A \g{line directive} is a \g{singular line directive} if it is only ever active for a single \g{line}. If it is matched on two consecutive \gpl{line} this results in two separate \gpl{resulting textual component}. \\

When a \g{singular line directive} is \glink{processing}{processed}, processing begins anew over the \g{content binding} until the end of the \g{line} is reached, at which point the \g{resulting textual component} is ended. After that, control is handed back to the \g{standard processing loop}.

\definesubsubsection{Spanning Line Directives}
A \g{line directive} is a \g{spanning line directive} if the \g{identifier} contains a \g{content binding}, and if \glink{match}{matches} on consecutive \gpl{line} of the \g{identifier} are interpreted as a single \g{match}. The semantics of such a spanning match are as follows: Only a single \g{resulting textual component} is produced for all the consecutively \glink{match}{matching} \gpl{line}. The \g{text} of this \g{resulting textual component} is produced by concatenating the contents of the \g{content binding} on each \g{line}. If the \g{content binding} does not \g{match} the \g{newline} on every \g{line}, the \g{newline} must be inserted between each \g{string} of the \g{content binding}. \\

When a \g{spanning line directive} is \glink{processing}{processed}, processing begins anew over the \g{content binding} until the end of the \g{line} is reached. Standard end of \g{line} \g{interpretation} proceeds. If the following \g{line} \glink{match}{matches} the same \g{spanning line directive} as before, processing begins anew over the \g{content binding} thereof without any new \gpl{resulting textual component} being started or inserted. If the following \g{line} does not \g{match} the same \g{spanning line directive} as before, the \g{resulting textual component} is ended and control is handed back to the \g{standard processing loop}.

\definesubsubsection{Guarded Line Directives}
A \g{line directive} is a \g{guarded line directive} if its \glink{match}{matched} region is specified by two \gpl{identifier} that each match a single \g{line}. The \g{text} of the \g{resulting textual component} is the \g{text} from the \g{line} immediately after the \g{line} the first \g{identifier} \glink{match}{matches} until and including the \g{line} immediately before the \g{line} the second \g{identifier} \glink{match}{matches}. \\

When a \g{guarded line directive} is \glink{processing}{processed}, processing begins anew over the \g{content binding} until the the part of the \g{identifier} after the \g{content binding} is \glink{full match}{fully matched}, at which point the \g{resulting textual component} is ended and control is handed back to the \g{standard processing loop}.

\definesubsection{Paragraph}
\begin{identifier}{paragraph}
<spaces [ ]*><content![ ].*>
\end{identifier}
\definetextualcomponent{paragraph}{margin: top, bottom} \\

The paragraph can only be \glink{match}{matched} if no other \g{line directive} \glink{match}{matches}. \Gpl{line} belong to the same paragraph until the length of \inline{spaces} changes, a new \g{inline directive} is recognised, or an \g{empty line} is encountered. The paragraph is a \g{spanning line directive}. \\

Paragraphs are visually distinguished by a margin above and below the \g{text}. An \g{implementation} may additionally employ indentation rules to distinguish the beginning of a paragraph. \\

\begin{examples}
  \begin{examplesource}
This is a paragraph
that spans multiple lines

This is another paragraph.
  \end{examplesource}
  \begin{exampleoutput}
    This is a paragraph\\
    that spans multiple lines.\\
    \\
    This is another paragraph.
  \end{exampleoutput}
  \begin{examplesource}
Paragraph One
  Paragraph Two
  \end{examplesource}
  \begin{exampleoutput}
    Paragaph One\\
    \\
    Paragraph Two
  \end{exampleoutput}
\end{examples}

%%% Local Variables:
%%% mode: latex
%%% TeX-master: "markless"
%%% TeX-engine: luatex
%%% TeX-command-extra-options: "-shell-escape"
%%% End:

\definesubsection{Blockquote}
\begin{identifier}{blockquote header}
\~ <content .+>
\end{identifier}
\begin{identifier}{blockquote body}
| <content .*>
\end{identifier}
\definetextualcomponent{blockquote header}{margin: left; font-weight: bold}
\definetextualcomponent{blockquote body}{margin: left} \\

The blockquote header is a \g{singular line directive} that identifies the source of a quote. Only the \g{text} held by the \g{content binding} is outputted into the \g{resulting textual component}. The blockquote header can only contain \gpl{inline directive}. \\

The blockquote body is a \g{spanning line directive} that identifies a body of \g{text} that is being quoted. The blockquote body can contain any \g{directive} with the condition that the \gpl{directive} are matched against the \g{text} of the \g{resulting textual component}. \\

An implementation may choose to group the \comp{blockquote header} and \comp{blockquote body} together and reorder them if they are found consecutive to one another. However, a body can only ever be grouped together with a single header. In the case where a header lies between two bodies, the header is counted to belong to the second body. If a header is found without a corresponding body, the \g{implementation} may \glink{signalling}{signal} a \g{warning}. \\

\begin{examples}
  \begin{examplesource}
~ This Document
| The blockquote header is a \
| singular line directive.
  \end{examplesource}
  \begin{exampleoutput}
    \begin{blockquote}[This Document]
      The blockquote header is a singular line directive.
    \end{blockquote}
  \end{exampleoutput}
  \begin{examplesource}
| Unattributed text.
  \end{examplesource}
  \begin{exampleoutput}
    \begin{blockquote}
      Unattributed text.
    \end{blockquote}
  \end{exampleoutput}
\end{examples}

%%% Local Variables:
%%% mode: latex
%%% TeX-master: "markless"
%%% TeX-engine: luatex
%%% End:

\definesubsection{Lists}
\begin{identifier}{Ordered List}
<number ~d+> <content .*>
(<spacing ~_+> <content .*>)*
\end{identifier}
\begin{identifier}{Unordered List}
- <content .*>
(<spacing ~_+> <content .*>)*
\end{identifier}
\definetextualcomponent{ordered list}{margin: left}
\definetextualcomponent{ordered list item}{display: list-item; list-item-prefix: number}
\definetextualcomponent{unordered list}{margin: left}
\definetextualcomponent{unordered list item}{display: list-item; list-item-prefix: dot} \\

The lists are \gpl{spanning line directive} and mark the enumeration of one or more items of a list. They can contain any \g{directive} with the condition that the \gpl{directive} are matched against the \g{text} of the \g{resulting textual component}. \\

After the respective list \g{identifier} has been \glink{match}{matched}, a new respective item \g{textual component} in which the higher \g{level} \g{text} is contained, is inserted for each \g{match} into the spanning \g{resulting textual component}. A single \g{match} may span over multiple \gpl{line} if the \g{text} \glink{match}{matched} by the \inline$spacing$ \g{binding} is of the same length as that of the \inline$number$ \g{binding}. In such a case, each item \g{match} itself is treated like a \g{spanning line directive} where the \g{content binding} is concatenated. \\

Ordered list items must be numbered by the \g{decimal number} given by the \inline$number$ \g{binding}, even if there is no order to how the numbers appear in the list or if there are duplicates. \\

\begin{examples}
\begin{examplesource}
- Finish this spec
- Implement a parser
\end{examplesource}
  \begin{exampleoutput}
    \begin{minipage}{0.5\textwidth}
      \begin{itemize}[noitemsep]
      \item Finish this spec
      \item Implement a parser
      \end{itemize}
    \end{minipage}
  \end{exampleoutput}
\begin{examplesource}
1 Buy some ingredients
2 Clean the kitchen
  Don't forget the sink!
5 Watch TV
\end{examplesource}
  \begin{exampleoutput}
    \begin{minipage}{0.5\textwidth}
      \begin{enumerate}[noitemsep]
      \item Buy some ingredients
      \item Clean the kitchen\\Don't forget the sink!
        \setcounter{enumi}{4}
      \item Watch TV
      \end{enumerate}
    \end{minipage}
  \end{exampleoutput}
\end{examples}

%%% Local Variables:
%%% mode: latex
%%% TeX-master: "markless"
%%% TeX-engine: luatex
%%% TeX-command-extra-options: "-shell-escape"
%%% End:

\definesubsection{Header}
\begin{identifier}{header}
<level #+> <content .+>
\end{identifier}
\definetextualcomponent{header}{font-weight:bold; font-size: 1-level; indent: true; label: content} \\

The header is a \g{singular line directive}. It represents a section heading. Only the \g{text} held by the \g{content binding} is outputted to the \g{resulting textual component}. The header can only contain \gpl{inline directive}.\\

The length of the \inline$level$ \g{binding} determines the level of the heading. The level may potentially be infinitely high, though the \g{implementation} may represent levels above a certain number in the same manner. It must however support a different representation for at least levels 1 and 2. Generally, the higher the level, the smaller the font size of the heading should be. \\

An \g{implementation} may choose to number each header, where this number prefix is put together by the number prefix of the header on a level one higher followed by a dot and a counter representing how many headers of the same level have appeared until and including the current one since the last header of a higher level. In the case of a level one heading only the counter is used, as there is no higher level prefix to prepend. In the case where no level one higher is contained in the \g{document}, the level is treated as if it existed with the counter for it being 0.\\

The \g{resulting textual component} is associated with a \g{label} of the same name as the \g{text} of the \g{resulting textual component}. \\

\begin{examples}
  \begin{examplesource}
# Header
The header is a singular line
directive
## Subsection
That allows neat sectioning!
  \end{examplesource}
  \begin{exampleoutput}
    \textbf{\quad\Large Header}\\
    The header is a singular line\\
    directive.\\
    \textbf{\quad\large Subsection}\\
    That allows neat sectioning!\\
  \end{exampleoutput}
\begin{examplesource}
# Cooking a Lasagna
Here's what you have to buy:
## Ingredients
A buncha stuff!
## Steps
It's a lengthy recipe, but finally \
you'll have to
#### Bake it
\end{examplesource}
  \begin{exampleoutput}
    \textbf{\quad\Large 1 Cooking a Lasagna}\\
    Here's what you have to buy: \\
    \textbf{\quad\large 1.1 Ingredients}\\
    A buncha stuff! \\
    \textbf{\quad\large 1.2 Steps}\\
    It's a lengthy recipe, but finally you'll have to \\
    \textbf{\quad\small 1.2.0.1 Bake it}\\
  \end{exampleoutput}
\end{examples}

%%% Local Variables:
%%% mode: latex
%%% TeX-master: "markless"
%%% TeX-engine: luatex
%%% TeX-command-extra-options: "-shell-escape"
%%% End:
\definesubsection{Horizontal Rule}
\begin{identifier}{horizontal-rule}
==+
\end{identifier}
\definetextualcomponent{horizontal-rule}{display: line} \\

The horizontal rule is a \g{singular line directive}. It is translated into a \g{resulting textual component} that represents a horizontal rule or break on the page. This must span the entire width of the document and could be represented by a thin line. If the \g{document} cannot support the drawing of lines, the horizontal rule may instead be approximated through other means.\\

\begin{examples}
  \example{==}{\rule{0.5\textwidth}{1pt}}
  \begin{examplesource}
And now, for a brief break.
=====
Back to the show!
  \end{examplesource}
  \begin{exampleoutput}
    And now, for a brief break. \\
    \rule{0.5\textwidth}{1pt} \\
    Back to the show!
  \end{exampleoutput}
\end{examples}

%%% Local Variables:
%%% mode: latex
%%% TeX-master: "markless"
%%% TeX-engine: luatex
%%% End:
\definesubsection{Code block}
\begin{identifier}{code block}
<prefix ::+><language ![,]+>?<options .*>
<content .*>
<prefix>
\end{identifier}
\definetextualcomponent{code block}{font-family: monospace; white-space: preserve} \\

The code block is a \g{guarded line directive}. It marks the \g{text} to belong to a \g{textual component} that somehow distinguishes the block as source code. Only the \g{text} held by the \g{content binding} is outputted to the \g{resulting textual component}. The code block \g{directive} cannot contain any other \gpl{directive}. \\

The \gpl{newline} and \g{whitespace} must be represented exactly as in the source text. Multiple consecutive \g{whitespace} \gpl{character} cannot be combined and must be individually represented. A \g{newline} \g{character} cannot be escaped and must always result in a new line being started. \G{escaping} is deactivated within the content, meaning backslashes are output literally in the \g{resulting textual component}.  \\

The \inline$options$ \g{binding} holds potential parameters that can configure the \g{style} of the \g{resulting textual component}. The syntax and effect of the options is \g{implementation dependant}. If a language is requested in the \linline$language$ \g{binding} that the \g{implementation} does not have specific support for, a \g{warning} is \glink{signalling}{signalled}. The language \inline$text$ must always be supported and will incur no styling changes. \\

\begin{examples}
\begin{examplesource}
Some unexciting code:
:: common-lisp
(print "Hello world")
::
\end{examplesource}
  &$\Rightarrow$&
  \begin{tabular}{@{}l@{}}
Some unexciting code: \\
\begin{lstlisting}[style=codestyle,language=Lisp,showstringspaces=false]
(print "Hello world")
\end{lstlisting}
\end{tabular}
\end{examples}

%%% Local Variables:
%%% mode: latex
%%% TeX-master: "markless"
%%% TeX-engine: luatex
%%% TeX-command-extra-options: "-shell-escape"
%%% End:
\definesubsection{Instruction}
\begin{identifier}{instruction}
\! <instruction .*>
\end{identifier}

The instruction is a \g{singular line directive}. Its purpose is to interact with the \g{implementation} and cause it to perform differently. There is no corresponding \g{resulting textual component} for the instruction \g{directive}. \\

An \g{implementation} is allowed to add further instructions. If an instruction is not recognised, the \g{implementation} must \glink{signalling}{signal} an \g{error}. \\

\definesubsubsection{Set}
\defineinstruction{set}{set <variable ![ ]+> <value .+>} \\

Sets the state of the \g{variable} of the given name to a certain value. An \g{implementation} may check the value for validity and \glink{signalling}{signal} an \g{error} if it is invalid.  An \g{implementation} is allowed to add further variables. If a variable is not recognised, the \g{implementation} must \glink{signalling}{signal} an \g{error}. \\

\definesubsubsubsection{Line Break Mode}
\definevariable{line-break-mode}{show} \\

The \var{line-break-mode} variable may only assume two values: \inline$show$, and \inline$hide$. If the line break mode is \inline$show$, when the processor encounters an unescaped \g{newline}, a new \g{line} is started in the output \g{document}. \\

\begin{examples}
\begin{examplesource}
! set line-break-mode show
foo
bar\
baz
! set line-break-mode hide
bada
boom
\end{examplesource}
  \begin{exampleoutput}
    foo\\
    barbaz\\
    badaboom
  \end{exampleoutput}
\end{examples}

%%% Local Variables:
%%% mode: latex
%%% TeX-master: "markless"
%%% TeX-engine: luatex
%%% TeX-command-extra-options: "-shell-escape"
%%% End:
\definesubsubsubsection{Metadata}
\definevariable{author}{} \\
\definevariable{copyright}{} \\

Declares \g{metadata} about the \g{document}. The \g{implementation} may use this information and embed it into the output \g{document}.

%%% Local Variables:
%%% mode: latex
%%% TeX-master: "markless"
%%% TeX-engine: luatex
%%% TeX-command-extra-options: "-shell-escape"
%%% End:

%%% Local Variables:
%%% mode: latex
%%% TeX-master: "markless"
%%% TeX-engine: luatex
%%% TeX-command-extra-options: "-shell-escape"
%%% End:
\definesubsubsection{Message}
\defineinstruction{info}{info <message .*>}
\defineinstruction{warn}{warn <message .*>}
\defineinstruction{error}{error <message .*>} \\

The \instr{info} instruction causes the \g{implementation} to \glink{signalling}{signal} the given message.
The \instr{warn} instruction causes the \g{implementation} to \glink{signalling}{signal} a \g{warning} with the given message.
The \instr{error} instruction causes the \g{implementation} to \glink{signalling}{signal} an \g{error} with the given message. \\

%%% Local Variables:
%%% mode: latex
%%% TeX-master: "markless"
%%% TeX-engine: luatex
%%% TeX-command-extra-options: "-shell-escape"
%%% End:
\definesubsubsection{Include}
\defineinstruction{include}{include <file .*>} \\

Causes the implementation to \glink{interpretation}{interpret} the contents of the given file. If the file is not accessible for some reason, the \g{implementation} must \glink{signalling}{signal} an \g{error}. \\

%%% Local Variables:
%%% mode: latex
%%% TeX-master: "markless"
%%% TeX-engine: luatex
%%% TeX-command-extra-options: "-shell-escape"
%%% End:
\definesubsubsection{Directives}
\defineinstruction{disable}{disable <directive ![ ]+>( <directive ![ ]+>)*}
\defineinstruction{enable}{enable <directive ![ ]+>( <directive ![ ]+>)*} \\

The \instr{disable} and \instr{enable} instructions cause the \g{implementation} to respectively disable or enable the named \gpl{directive}. The name of a directive must be recognised regardless of the case the user writes the directive in. If a given name is not recognised, the \g{implementation} may \glink{signalling}{signal} a \g{warning}. If a user tries to disable the \comp{paragraph} \g{directive}, an \g{error} must be \glink{signalling}{signalled}. \\

\begin{examples}
\begin{examplesource}
! disable instruction
! error Exit!
\end{examplesource}
  \begin{exampleoutput}
    ! error Exit!
  \end{exampleoutput}
\end{examples}

%%% Local Variables:
%%% mode: latex
%%% TeX-master: "markless"
%%% TeX-engine: luatex
%%% TeX-command-extra-options: "-shell-escape"
%%% End:


%%% Local Variables:
%%% mode: latex
%%% TeX-master: "markless"
%%% TeX-engine: luatex
%%% TeX-command-extra-options: "-shell-escape"
%%% End:
\input{5.8-comment.tex}
\definesubsection{Embed}
\begin{identifier}{embed}
\[<type ![ ]*> <target ![ ]*>( <parameter ![ ]*>)*\]
\end{identifier}
\definetextualcomponent{embed}{display: block;target: target} \\

The embed is a \g{singular line directive}. The content of the \inline$type$ binding determines the embed's type, and the \inline$parameter$ bindings determine the embed's parameters. The style of the \g{resulting textual component} is dynamically dependant on the given type. If a type is specified that the \g{implementation} does not recognise, a \g{warning} is \glink{signalling}{signalled} and no \g{resulting textual component} is outputted into the \g{document}. \\

Unless the \ident{embed-property-width} or \ident{embed-property-height} parameters are present, the size of the embed \g{resulting textual component} is constrained to be smaller than the width and height of the \g{document} while preserving the embed content's aspect ratio. If the \g{document} has no width or height, or the embed content's dimensions are smaller than both of those, then the embed content is sized to its own dimensions. \\

\definesubsubsection{Embed Types}
An \g{implementation} must at least support the types specified in this section if permitted by the output \g{document}, but may add additional options the implications of which are completely \g{implementation dependant}. If the output \g{document} does not support a particular type, a \comp{paragraph} containing a single \comp{url} \g{textual component} is outputted with its target and content set to the \inline$target$ \g{binding}'s value and a \g{warning} is \glink{signalling}{signalled}. If the \g{implementation} does not support the requested type at all, an \g{error} is \glink{signalling}{signalled}. \\

\definesubsubsubsection{Image}
\defineidentifier{embed-type-image}{image}
\definestyle{embed-type-image}{interaction: image} \\

Embeds the image pointed to by the \inline$target$ into the document. The supported image formats are \g{implementation dependant}. If the format of the target is not supported by the \g{implementation}, the \g{directive} is treated as if it were given an unknown type. \\

\begin{examples}
  \example{[image markless-logo.png]}{\includegraphics[scale=0.5]{markless-logo}}
\end{examples}

%%% Local Variables:
%%% mode: latex
%%% TeX-master: "markless"
%%% TeX-engine: luatex
%%% End:

\definesubsubsubsection{Video}
\defineidentifier{embed-type-video}{video}
\definestyle{embed-type-video}{interaction: video;target: target} \\

\begin{examples}
  \example{[video sample.mp4]}{\url{file://./sample.mp4}}
\end{examples}
%%% Local Variables:
%%% mode: latex
%%% TeX-master: "0-markless"
%%% TeX-engine: luatex
%%% End:

\definesubsubsubsection{Audio}
\defineidentifier{embed-type-audio}{audio}
\defineidentifier{embed-property-loop}{loop}
\defineidentifier{embed-property-autoplay}{autoplay}
\definestyle{embed-type-audio}{interaction: audio} \\

Embeds the audio file pointed to by the \inline$target$ into the document. The supported audio formats are \g{implementation dependant}. If the format of the target is not supported by the \g{implementation}, the \g{directive} is treated as if it were given an unknown type. The \g{resulting textual component} must be interactive in such a way that the \g{user} is presented with a way to start, pause, seek, and change the volume of the audio. The audio track should not play automatically, unless the \ident{embed-property-autoplay} flag property is present. If the \ident{embed-property-loop} flag property is present, the audio track should start over from the beginning once it reaches the end. Since an audio file does not have any dimensions associated with it, the \g{implementation} is free to choose the sizing it deems appropriate. \\

\begin{examples}
  \example{[audio sample.mp3]}{\url{file://./sample.mp3}}
\end{examples}
%%% Local Variables:
%%% mode: latex
%%% TeX-master: "markless"
%%% TeX-engine: luatex
%%% End:

\definesubsubsubsection{Source}
\defineidentifier{embed-type-source}{source}
\defineidentifier{embed-property-options}{options <options .+>}
\defineidentifier{embed-property-language}{language <language .+>}
\defineidentifier{embed-property-start}{start <start ~n+>}
\defineidentifier{embed-property-end}{end <end [+]?~n+>}
\defineidentifier{embed-property-encoding}{encoding <encoding .+>}
\definestyle{embed-type-source}{font-family: monospace; white-space: preserve} \\

Embeds the source code pointed to by the \inline$target$ into the document. To do this, the file is read as a text file in the \g{encoding} specified by \ident{embed-property-encoding}. If the \ident{embed-property-encoding} is not given, UTF-8 \g{encoding} is assumed. If an encoding is requested that the \g{implementation} does not support, an \g{error} is \glink{signalling}{signalled}.\\

The file's contents are split into a sequence of lines. If \ident{embed-property-start} is given, as many lines as indicated in its \inline$start$ \g{binding} are discarded from the front. If \ident{embed-property-end} is given, and its \inline$end$ \g{binding} starts with a \unicode{2B}, as many lines as indicated in the \g{binding} are output into the \g{resulting textual component}. If \ident{embed-property-end} is given, but its \g{binding} does not start with \unicode{2B}, the lines until and including the line indicated by the \inline$end$ \g{binding} are output into the \g{resulting textual component}, counting the first line read from the file as the line numbered \inline$1$. \\

A line in this context is determined as follows: each line in the file is delimited by either the beginning of the file, the end of the file, or the nearest Linefeed \unicode{A} characters. This means that unlike a Markless \g{line}, the Linefeed \unicode{A} end of line marker cannot be escaped. \\

The \ident{embed-property-language} and \ident{embed-property-options} options hold parameters that configure the \g{style} of the \g{resulting textual component}. The syntax and effect of the options is \g{implementation dependant}, but it must be the same as for the \ident{code block} \g{directive}. If a language is requested that the \g{implementation} does not have specific support for, a \g{warning} is \glink{signalling}{signalled}. \\

\begin{examples}
  \begin{examplesource}
    [ source 5.9.1.2-source.tex, language tex, end 2 ]
  \end{examplesource}
  &$\Rightarrow$&\\
  \begin{tabular}{@{}l@{}}
\begin{lstlisting}[style=codestyle,showstringspaces=false]
\definesubsubsubsection{Source}
\defineidentifier{embed-type-source}{source}
\end{lstlisting}
  \end{tabular}
\end{examples}


%%% Local Variables:
%%% mode: latex
%%% TeX-master: "markless"
%%% TeX-engine: luatex
%%% TeX-command-extra-options: "-shell-escape"
%%% End:


%%% Local Variables:
%%% mode: latex
%%% TeX-master: "markless"
%%% TeX-engine: luatex
%%% TeX-command-extra-options: "-shell-escape"
%%% End:

\definesubsubsection{Embed Parameters}
The parameters are processed in the order they are given and can effect both the content of the \g{resulting textual component} as well as its \g{style}. A parameter may also affect the processing of parameters after it. Two general types of parameters are defined: flag parameters and value parameters. Flag parameters are single parameters that add or remove an attribute from the \g{resulting textual component}'s \g{style}. Value parameters add an attribute whose value is determined by the parameter following the current one. The following parameter is then skipped over and thus not processed. \\

An \g{implementation} must at least support the parameters specified in this section if permitted by the output \g{document}, but may add additional parameters the implications of which are completely \g{implementation dependant}. If the output \g{document} does not support a particular parameter, or an unknown parameter is given, a \g{warning} is \glink{signalling}{signalled}. \\

\definesubsubsubsection{Float}
\defineidentifier{embed-property-float}{float <orientation left|right>}
\definestyle{embed-property-float}{float: orientation} \\

Causes the embed to float on either the left or right side of the \g{document}. All the \gpl{resulting textual component} after it will flow around it.

%%% Local Variables:
%%% mode: latex
%%% TeX-master: "markless"
%%% TeX-engine: luatex
%%% TeX-command-extra-options: "-shell-escape"
%%% End:

\definesubsubsubsection{Width}
\defineidentifier{embed-property-width}{width (<pixels ~n+px>|<percent ~n+\%>)}
\definestyle{embed-property-width}{width: size} \\

Causes the embed content's width to be fixed to the specified size. The size can be given in either \inline$pixels$ or \inline$percent$ where \inline$pixels$ will set the width to be the exact amount of pixels given if the document is viewed at its native resolution. \inline$percent$ will scale the width to the given percentage of the width of the \g{document}. If the \g{document} should not have a width, the \inline$percent$ specification does nothing. Unless the \ident{embed-property-height} is also specified, the embed content's aspect ratio must be preserved. \\

\begin{examples}
  \example{[image markless-logo.png width 50px]}{\includegraphics[width=50px]{markless-logo}}
\end{examples}

%%% Local Variables:
%%% mode: latex
%%% TeX-master: "markless"
%%% TeX-engine: luatex
%%% End:

\definesubsubsubsection{Height}
\defineidentifier{embed-property-height}{height (<pixels ~n+px>|<percent ~n+\%>)}
\definestyle{embed-property-height}{height: size} \\

Causes the embed content's height to be fixed to the specified size. The size can be given in either \inline$pixels$ or \inline$percent$ where \inline$pixels$ will set the height to be the exact amount of pixels given if the document is viewed at its native resolution. \inline$percent$ will scale the height to the given percentage of the height of the \g{document}. If the \g{document} should not have a height, the \inline$percent$ specification does nothing. Unless the \ident{embed-property-width} is also specified, the embed content's aspect ratio must be preserved. \\

\begin{examples}
  \example{[ image markless-logo.png width 50px height 100px ]}{\includegraphics[width=50px,height=100px]{markless-logo}}
\end{examples}

%%% Local Variables:
%%% mode: latex
%%% TeX-master: "markless"
%%% TeX-engine: luatex
%%% TeX-command-extra-options: "-shell-escape"
%%% End:


%%% Local Variables:
%%% mode: latex
%%% TeX-master: "markless"
%%% TeX-engine: luatex
%%% TeX-command-extra-options: "-shell-escape"
%%% End:


%%% Local Variables:
%%% mode: latex
%%% TeX-master: "markless"
%%% TeX-engine: luatex
%%% End:


%%% Local Variables:
%%% mode: latex
%%% TeX-master: "markless"
%%% TeX-engine: luatex
%%% End:
