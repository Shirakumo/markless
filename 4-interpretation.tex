\definesection{Interpretation}

\warn{This section is unfinished!}

\definesubsection{Parser State}
An \g{implementation} must keep several parts of information that can be modified by the \g{interpretation}.

\begin{itemize}
\item A \g{document} \g{textual component} where all resulting \g{text} is kept.
\item The \g{position} within the resulting \g{document}.
\item The \g{cursor} within the inputting \g{document}.
\item The \g{line break mode}.
\item A list of \gpl{disabled directive}.
\item A table associating \gpl{label} to \gpl{textual component}. Label names must be \g{case insensitive}.
\item Additionally, it might keep a variety of internal variables that can be changed by an \g{instruction}.
\end{itemize}

\definesubsection{Parser Steps}
When \glink{interpretation}{interpreting} a \g{document} exactly three things can happen to modify the resulting \g{document}:

\begin{itemize}
\item A \g{textual component} is started. This causes a new instance of the component to be inserted at the current \g{position}, the \g{position} to be moved to inside this instance, and the \g{level} to be increased by one.
\item An open \g{textual component} is ended. If this component is not the uppermost, this step is repeated for the \g{current component} until the component to close is the current. The current \g{position} is moved to outside and after the component and the \g{level} is decreased by one.
\item \G{text} is inserted. This means that the \g{text} is inserted at the current \g{position} and the \g{position} is moved to after the inserted \g{text}.
\end{itemize}



% Line break behaviour
% Backslash escaping
% Parsing order
% Labels
% Parser states and switches

%%% Local Variables:
%%% mode: latex
%%% TeX-master: "0-markless"
%%% TeX-engine: luatex
%%% End:
