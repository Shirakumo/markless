\definesection{Interpretation}
This section describes the procedure by which an \g{implementation} \glink{interpretation}{interprets} a \g{document}. This procedure is used as a reference to allow verification of correctness. An \g{implementation} does not necessarily have to follow this procedure as long as the output it produces is equivalent with an \g{implementation} that does.

\definesubsection{State}
The following state is kept and updated as the procedure advances.

\begin{itemize}
\item The input stream from which characters are read.
\item The parser state variables such as the \g{line break mode}.
\item A \g{cursor}.
\item A stack wherein each entry is composed of a \g{directive} and a \g{textual component}.
\item A list of \gpl{disabled directive}.
\item A table associating \gpl{label} to \gpl{textual component}.
\end{itemize}

\definesubsection{Procedure}
\begin{step}
\item A ``root-directive'' and a ``root-component'' are pushed onto the stack.
\item If the input stream has things to read:
  \begin{step}
  \item A \g{line} is read from the input stream.
  \item The \g{cursor} is set to the beginning of the \g{line}.
  \item The stack is traversed upwards from the bottom:
    \begin{step}
    \item The \g{directive} at the current stack entry attempts to \g{match}.
    \item If the \g{match} succeeds:
      \begin{step}
      \item The \g{cursor} is advanced by the \glink{match}{matched} \gpl{character}.
      \item The current stack entry is advanced upwards.
      \item Go to 2.3.1.
      \end{step}
    \item The stack is unwound down to and including the current stack entry. See \sectionref{Stack Unwinding}.
    \end{step}
  \item The \g{directive} on top of the stack is invoked:
    \begin{step}
    \item If an \g{applicable directive} \glink{match}{matches}:
      \begin{step}
      \item The \glink{match}{matched} \g{directive} may enter \gpl{textual component} into the \g{current component}.
      \item The \glink{match}{matched} \g{directive} may push itself and a \g{textual component} onto the stack or perform other changes to the state as specified.
      \item The \g{cursor} is advanced by the \glink{match}{matched} \gpl{character}.
      \item Go to 2.4.
      \end{step}
    \item The \g{character} at the \g{cursor} is added to the \g{current component}.
    \item The \g{cursor} is advanced by the \g{character}.
    \end{step}
  \item If the \g{cursor} is not yet at the end of the \g{line}:
    \begin{step}
    \item Go to 2.4.
    \end{step}
  \item If the \g{line break mode} is \inline$show$, a \g{newline} is added to the \g{current component}.
  \item Go to 2.
  \end{step}
\item The stack is unwound fully. See \sectionref{Stack Unwinding}.
\item The interpretation is complete. The ``root-component'' represents the resulting \g{document}.
\end{step}

\definesubsubsection{Stack Unwinding}
\begin{step}
\item If the stack is taller than the desired height:
  \begin{step}
  \item If the directive on top of the stack is a \g{inline directive}:
    \begin{step}
    \item The \g{current component} is converted to one that has no \g{style}.
    \item \Gpl{character} that have been consumed by the prefix match of the \g{directive} are prepended to the \g{current component}.
    \item Other potentially necessary actions to undo the match of the \g{directive} are performed.
    \end{step}
  \item The top of the stack is popped off.
  \item Go to 1.
  \end{step}
\end{step}

\definesubsubsection{Root Directive}
The root directive always \glink{match}{matches} but is never considered an \g{applicable directive}.

%%% Local Variables:
%%% mode: latex
%%% TeX-master: "markless"
%%% TeX-engine: luatex
%%% TeX-command-extra-options: "-shell-escape"
%%% End:
